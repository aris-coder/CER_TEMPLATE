\section{Références des méthodes et outils}
\begin{thebibliography}{9}
\setlength{\itemsep}{0pt}
\bibitem{histinfo} Wikipédia, \textit{Histoire de l'informatique}. \url{https://fr.wikipedia.org/wiki/Histoire_de_l%27informatique}.
\bibitem{gnu} Wikipédia, \textit{Projet GNU}. \url{https://fr.wikipedia.org/wiki/Projet_GNU}.
\bibitem{libre} Wikipédia, \textit{Histoire du logiciel libre}. \url{https://fr.wikipedia.org/wiki/Histoire_du_logiciel_libre}.
\bibitem{licence} Wikipédia, \textit{Licence de logiciel}. \url{https://fr.wikipedia.org/wiki/Licence_de_logiciel}.
\bibitem{cyberethique} Wikipédia, \textit{Éthique de l'informatique}. \url{https://fr.wikipedia.org/wiki/%C3%89thique_de_l%27informatique}.
\bibitem{dev} Wikipédia, \textit{Processus de développement logiciel}. \url{https://fr.wikipedia.org/wiki/Processus_de_d%C3%A9veloppement_logiciel}.
\bibitem{communication} Atlassian, \textit{Use Case : technical documentation}. \url{https://www.atlassian.com/fr/software/confluence/use-cases/technical-documentation}.
\bibitem{documentation} Wikipédia, \textit{Documentation logicielle}. \url{https://fr.wikipedia.org/wiki/Documentation_logicielle}.
\bibitem{overleaf} Overleaf, \textit{Overleaf}. \url{https://www.overleaf.com/}.
\end{thebibliography}

\newpage
\section{Références bibliographiques complémentaires}
\includegraphics[width=0.5\linewidth]{Images_bibliographiques/google_img.jpg}\hspace{1.5cm} \includegraphics[width=0.5\linewidth]{Images_bibliographiques/Wiki_img.jpg}
